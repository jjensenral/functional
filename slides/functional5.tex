% Created 2022-01-16 Sun 11:18
% Intended LaTeX compiler: pdflatex
\documentclass[presentation]{beamer}
\usepackage[utf8]{inputenc}
\usepackage[T1]{fontenc}
\usepackage{graphicx}
\usepackage{grffile}
\usepackage{longtable}
\usepackage{wrapfig}
\usepackage{rotating}
\usepackage[normalem]{ulem}
\usepackage{amsmath}
\usepackage{textcomp}
\usepackage{amssymb}
\usepackage{capt-of}
\usepackage{hyperref}
\usepackage{xypic}
\usepackage{tabularx}
\xyoption{pdf}
\usetheme{default}
\newcommand\cat[1]{\ensuremath{\mathcal{#1}}}
\newcommand\hm[2]{\hom(#1,#2)}
\newcommand\nat[2]{{#1}\overset{\cdot}{\rightarrow}{#2}}
\newcommand\tup[1]{\langle #1\rangle}
\newcommand\id[1]{\mathrm{id}_{#1}}
\author{jens.jensen@stfc.ac.uk \\0000-0003-4714-184X\\CC-BY 4.0}
\date{May 22, 2022}
\title{Functional Programming 5}
\hypersetup{
 pdfauthor={jens.jensen@stfc.ac.uk},
 pdftitle={Functional Programming 5},
 pdfkeywords={functional, monad, programming theory},
 pdfsubject={functional programming},
 pdfcreator={Emacs 27.1 (Org mode 9.3) then by hand}, 
 pdflang={English}}
\begin{document}

\maketitle
\begin{frame}{Outline of Talks}

  \begin{enumerate}
  \item Basics of functional programming
    \begin{itemize}
    \item Recursion, dividing problems into sub-cases, immutable variables
    \end{itemize}
  \item Mapping
  \item Lists and conses
    \begin{itemize}
    \item Basic structure: helper functions, naming
    \end{itemize}
  \item Lambdas, higher order functions
    \begin{itemize}
    \item list comprehension
    \item symbols
    \end{itemize}
  \end{enumerate}

Still written in the author's spare time!

Very much a personal perspective, and not following any particular textbook.  Using \emph{meditations} and \emph{exercises} -- solutions to all exercises given during the talks.

\end{frame}


\section{Introibo}
\label{sec:org0edb596}


\section{Advanced(ish) Functional Programming}
\begin{frame}{Common/Advanced(ish) Features of Functional Languages}
\label{sec:org95ddb74}

\begin{enumerate}
\item Lambda (anonymous (unnamed) functions) and \emph{currying}
\item List comprehension
\item Functions -- mutually recursive, higher order
\item Symbols
\item \textbf{Tail recursion}
\item Scope and extent (Lisp)
\item \textbf{Types and type inference}
\item \textbf{Branch-on-pattern-matching and guards}
\item Memoisation
\item Lazy evaluation types
\item Pipes (not the lazy kind) style composition
\begin{itemize}
\item  \(h(g(f(x)))\equiv\texttt{(h (g (f x)))}\equiv{}x\vert{f}\vert{g}\vert{h}\)
\end{itemize}
\item Monads: theoretical framework for types and computation
\item Applied monads: Maybe, Arrays
\item Bonus section for survivors of MonadLand: Lisp Hacking
\end{enumerate}

\end{frame}



\begin{frame}[fragile]{TRO 1 (Tail Recursion Optimisation)}
\label{sec:orgd96bf98}
\begin{verbatim}
(disassemble 'fact)
0	dup	  
1	constant  0
2	eqlsign	  
3	goto-if-nil 1
6	constant  1
7	return	  
8:1	dup	  
9	constant  fact
10	stack-ref 2
11	sub1	  
12	call	  1
13	mult	  
14	return	  
\end{verbatim}

\end{frame}

\begin{frame}[fragile]{TRO 2}
\label{sec:org52905b9}
\begin{verbatim}
(defun fact (k)
  (if (zerop k) 1 (* k (fact (1- k)))))
\end{verbatim}
uses tail recursion but is not optimisable; this one is:
\begin{verbatim}
(defun fact (k result)
  (if (zerop k) result
    (fact (1- k) (* k result))))
\end{verbatim}

By the time the function calls itself, the first version has to remember \texttt{(* k ...)}; the second version doesn't need to remember anything.  Compilers can optimise the recursion call away and make it a loop (though it becomes harder to debug as the extra stack frame is missing).

\medskip
EL does not do \emph{tail recursion optimisation} (TRO) but it is an important technique in functional programming (and some day EL may do it too)
\end{frame}

\begin{frame}[fragile]{TRO 3}
\label{sec:org97c2a55}

\texttt{fact} has an extra parameter, so we need to call \texttt{(fact 12 1)}.  Here is \texttt{iota} from before in the same style (note, no \texttt{nreverse} -- meditate on why it is not needed):

\begin{verbatim}
(defun iota (k result)
  (if (zerop k) result (iota (1- k) (cons k result))))
\end{verbatim}
Notice it generates a list of $1,\ldots,k$; sometimes \texttt{(iota k)} is expected to generate $0,\ldots,k-1$

One way to fix the extra parameter is to make \texttt{result} optional.  Unfortunately, ELisp does not have default values for optional parameters.

\begin{verbatim}
(defun fact (k &optional result)
  (if (zerop k) (or result 1)
    (fact (1- k) (* k (or result 1)))))
fact
(fact 12)
479001600
\end{verbatim}

\end{frame}

\begin{frame}[fragile]{TRO 4}
\label{sec:orgc4e4ea1}

This would also do the trick, using a local recursive function defined with \texttt{labels} to do the actual work:

\begin{verbatim}
(defun fact (k)
  (labels ((fact1 (k result)
            (if (zerop k) 1 (fact1 (1- k) (* k result)))))
          (fact1 k 1)))
\end{verbatim}

Creating a helper function (scoped to the inside of \texttt{fact}) which can call itself recursively (hence using \texttt{labels} instead of \texttt{flet}).  Now \texttt{fact1} is TRO-able and the recursion will be optimised as a loop.
\end{frame}
\begin{frame}[fragile]{TRO 5}
\label{sec:orgc4e4ea1}

Using a wrapper function could also help with type checks:

\begin{verbatim}
(defun fact (k)
  (unless (and (integerp k) (>= k 0))
    (error "Improper argument to fact: %s" k))
  (labels ((fact1 (k result)
            (if (zerop k) 1 (fact1 (1- k) (* k result)))))
          (fact1 k 1)))
\end{verbatim}

As we have seen before, this takes the check out of the loop.  Previously we used a standalone helper function for \texttt{fact}, but in principle a section of code could have local optimisation parameters (at least in CL), allowing for extra optimisation of the helper function as before.

\end{frame}

\begin{frame}[fragile]{The Same Length Question}
From Talk 1, a check that all sequences have the same length:
\begin{verbatim}
(defun check-same-length (lists)
  "For a list of sequences, check that they are all the same length"
  (if (endp (cdr lists))		; zero or one elts
      t
    (check-same-length-1 (length (first lists)) (rest lists))))
(defun check-same-length-1 (len lists)
  "Helper function for check-same-length: check that all seqs in lists have length len"
  (cond
   ((endp lists) t)
   ((/= (length (first lists)) len) nil)
   (t (check-same-length-1 len (rest lists)))))
(check-same-length '((1 2) [2 3] "AB"))
t
(check-same-length '((1 2) (2) (3 4)))
nil
\end{verbatim}
\end{frame}

\begin{frame}[fragile]{Types}
  Meditation: what is the difference between
\begin{verbatim}
(defun process-list (k)
  (if (null k) t
    (reduce #'bar (map 'list #'foo k))))
\end{verbatim}
and
\begin{verbatim}
(defun process-list (k)
  (if (endp k) t
    (reduce #'bar (map 'list #'foo k))))
\end{verbatim}
\medskip

Hint: think of \texttt{(process-list [1 2])}
\end{frame}

\begin{frame}[fragile]{Types}
  What about this version?
\begin{verbatim}
(defun process-list (k)
  (declare (type list k))
  (if (null k) nil
    (reduce #'bar (map 'list #'foo k))))
\end{verbatim}

Meditation: What does \texttt{declare} do?  Hint: consider the difference between:
\begin{itemize}
\item ELisp checks that the argument is a list when it's passed in
\item The programmer \emph{promises} that \texttt{k} is a list, and consequences are undefined if it's not
\item It's a hint to the compiler that what comes in is likely to be a list, and the compiler is free to ignore it
\item It's like a comment
\end{itemize}
Answer: next slide
\end{frame}

\begin{frame}[fragile]{Types}
  Types are important for two reasons:
  \begin{itemize}
  \item Correctness -- tracking types (at read/compile time) helps ensure that the program is correct
  \item Performance and optimisation -- the compiler can make optimisations if it knows the specific type (eg.\ of a sequence or sequence element type)
  \end{itemize}
  Types can be specified in two ways:
  \begin{itemize}
  \item By the programmer.  It is an error if a programmer-specified type does not match.
    \begin{itemize}
    \item In CL, the consequences of a type error depend on the circumstances and what the standard/implementation define
    \end{itemize}
  \item It can be inferred.
    \begin{itemize}
    \item For example, in the sexp \texttt{(+ (floor a 6) (mod b 12))},
      \begin{itemize}
      \item \texttt{a} must be an integer, \texttt{(floor a 6)} is an integer
      \item \texttt{b} must be an integer; \texttt{(mod b 12)} is of type \texttt{(and (integer 0 12) fixnum)} (or similar)
      \item \texttt{+} can specialise to an integer-only (and only two parameters)
      \end{itemize}
    \end{itemize}
  \end{itemize}

\end{frame}

\begin{frame}[fragile]{Types}
  Functional languages support type inference.  For example, this will not work in ML-based languages:
\begin{verbatim}
let a = [2; (3,1); "ada"]
in ...
\end{verbatim}
as the language will try to infer a list-of-something but it sees a number, a pair and a string.

In contrast, Lisp will cheerfully accept
\begin{verbatim}
(let ((a '(2 (3 . 1) "ada")))
  ...)
\end{verbatim}

Although Emacs's type system is simpler it is worth learning (some of) CL's, partly for understanding functional programming as a paradigm, partly to uderstand Emacs' better.
  
\end{frame}

\begin{frame}[fragile]{Types and Inference}
  In a functional language like F\#, types are inferred (in Linux, run \texttt{fsharpi} to get the interpreter):
\begin{verbatim}
> [2;3;4] ;;
val it : int list = [2; 3; 4]
\end{verbatim}

Unlike arrays, in Lisp (E or C), we cannot directly specify the elements of a list:
\begin{verbatim}
(typep '(2 3 4) 'list)
t
\end{verbatim}
but in CL we can do
\begin{verbatim}
(defun intseqp (w) (every #'integerp w))
INTSEQP
(typep '(2 3 4) '(and list (satisfies intseqp)))
T
\end{verbatim}
\end{frame}

\begin{frame}[fragile]{Types}
  In Lisp, everything is part of a type hierarchy:
  \resizebox{\linewidth}{!}{
    \includegraphics{types.pdf}
  }

  This picture illustrates only the more important types.  Note four specialised float types: SBCL provides only two, with \texttt{long} and \texttt{double} being conflated, and \texttt{short} and \texttt{single} conflated.  In EL:
\begin{verbatim}
pi
3.141592653589793
(type-of pi)
float
\end{verbatim}
\end{frame}

\begin{frame}[fragile]{Types and Inference}
  In CL, types like \texttt{array} and \texttt{complex} (but not \texttt{list}) are compound types and can optionally specify their element types:
\begin{verbatim}
(type-of #(1 2 3))
(SIMPLE-VECTOR 3)
(typep #(1 2 3) '(array integer *))
T
\end{verbatim}
The \texttt{array} specifier is specialised with the type of element in the array and the dimensions.  The symbol \texttt{*} may be used as a short-hand for ``anything.''  Similarly, a rank-2 array could be specified with dimensions \texttt{(* *)}:
\begin{verbatim}
(typep #2A((1 2) (3 4)) '(array * (* *)))
T
\end{verbatim}

\medskip
Meditation: Note that \texttt{(array t)} is a proper subtype of \texttt{(array *)} -- why?

\end{frame}

\begin{frame}[fragile]{Types -- Generic Programming}
  Compare this with generic programming where the type is not known (until later):
\begin{verbatim}
let list_length x =
    let rec list_length1 x l =
        match x with
            | [] -> l
            | _ :: x1 -> list_length1 x1 (1+l)
    in list_length1 x 0
printfn "%d" (list_length [1;2;3])
printfn "%d" (list_length ["abra";"ca";"dabra"])
\end{verbatim}
The type of \texttt{list\_length} will be \texttt{'a list -> int}, meaning a list of a type \texttt{'a} mapping to an \texttt{int}.

\medskip
\texttt{map} (in F\#) would have type
\begin{verbatim}
('a -> 'b) -> 'a list -> 'b list
\end{verbatim}

\end{frame}
\begin{frame}[fragile]{Types -- Generic Programming}
  Suppose we wanted a list of mixed items: but F\# will refuse the following
\begin{verbatim}
[3;"abc"; 1.2]
\end{verbatim}
as the elements are three different types.  Instead we use a ``discriminated union'':
\begin{verbatim}
type Mixed =
    | String of string
    | Int of int
    | Float of float
printfn "%d" (list_length [Int 3; String "abc"; Float 1.2])
\end{verbatim}
\end{frame}
\begin{frame}[fragile]{Types -- Generic Programming}
  In C++, the types are managed in exactly the same way:
\begin{verbatim}
template<typename X>
std::size_t
list_length1(typename std::list<X>::const_iterator list,
             typename std::list<X>::const_iterator nil,
             std::size_t count)
{
    if(list == nil)
        return count;
    return list_length1<X>(++list, nil, 1+count);
}

template<typename X>
std::size_t
list_length(std::list<X> const &list)
{
    return list_length1<X>(list.cbegin(), list.cend(), 0);
}
\end{verbatim}
\end{frame}
\begin{frame}[fragile]{Types -- Generic Programming}
  In contrast, Lisp will (unless told otherwise) assume the most generic type:
  \begin{itemize}
  \item Lists are always assumed to be lists of \texttt{t}
  \item Arrays are assumed to be arrays of \texttt{t} (CL: unless spec'd)
  \end{itemize}
\begin{verbatim}
(defun list-length (l)
  (labels ((list-length1 (l c)
            (if (endp l) c
             (list-length1 (cdr l) (1+ c)))))
    (list-length1 l 0)))
list-length
(list-length '(3 "abc" 1.2))
3
\end{verbatim}
  This will happily work with any type (including user-defined types):
  \begin{itemize}
  \item The Lisp code is not duplicated for each type, like the C++ code is
  \item However, (as in C++), specialisations are possible in CLOS (CL only)
  \end{itemize}
\end{frame}

\begin{frame}[fragile]{Pattern Matching and Guards}
  A common example in (other) functional language is a sort of generalisation of \texttt{cond} (if such a thing can be imagined) where each case is a \emph{pattern} with an optional \emph{guard} -- this is a contrived example in F\#:
\begin{verbatim}
let rec func data sum =
    match data with
        | [] -> sum
        | (a,b) :: rest when a>b -> func rest (sum+a)
        | (_,b) :: rest -> func rest (sum+b)
        | _ -> failwith "could not parse data"
let test = [(3, 1) ; (4,1); (2,3) ; (-1, 2)]
in printfn "%d" (func test 0)
\end{verbatim}
When run, it prints 12.  Meditation: which simple refactoring would improve this function? (answer in a few slides)
\end{frame}

\begin{frame}[fragile]{Pattern Matching and Guards}
  The \texttt{Mixed} type from before can be deconstructed similarly:
\begin{verbatim}
type Mixed =
    | Float of float
    | String of string
    | Int of int

let explain (elt : Mixed) =
    match elt with
        | Float g -> printfn "Float %f" g
        | String s -> printfn "String %s" s
        | Int i -> printfn "Int %d" i

List.iter explain [Int 2; Float 2.71828; String "ada"]
\end{verbatim}
In Lisp the same effect can be achieved with \texttt{typecase} (which works like \texttt{case} but matches on the type)
\end{frame}

\begin{frame}[fragile]{Pattern Matching and Guards}
Lisp has the pattern matching function from macros' lambda lists as a standalone generalisation of \texttt{let}:
\begin{verbatim}
(destructuring-bind (a (b . c)) '(2 ((f g . y)))
  (list a b c))
(2 (f g . y) nil)
\end{verbatim}
This will raise an error:
\begin{verbatim}
(destructuring-bind (a (b c)) '(1 2) (list a b c))
\end{verbatim}
This means we can use it to bind variables in an expression, but not easily in a negative match.

\end{frame}

\begin{frame}[fragile]{Pattern Matching and Guards}
  Obviously we could write the function the conventional way (still without the helpful refactoring):
\begin{verbatim}
(defun func (data sum)
  (cond
   ((endp data) sum)
   ((> (caar data) (cdar data))
     (func (rest data) (+ sum (caar data))))
   ((and (consp (car data)) (atom (cdar data)))
     (func (rest data) (+ sum (cdar data))))
   (t (error "Failed to process ~S" data))))
\end{verbatim}
That third test is a bit hand-made (it tests for a cons cell).  Things to note:
\begin{itemize}
\item Unlike F\#, it doesn't make any bindings in the clauses
\item It doesn't pattern match in the sense that \texttt{destructuring-bind} does
\end{itemize}
\end{frame}

\begin{frame}[fragile]{Pattern Matching and Guards}
  Of course we could still write the pattern explicitly:
\begin{verbatim}
(cond
  ((endp data) sum)
  ((ignore-errors
      (destructuring-bind (a . b) (first data)) (> a b))
    (func (rest data) (+ sum a)))
  ...
\end{verbatim}
Here, \texttt{ignore-errors} will return \texttt{nil} if \texttt{destructuring-bind} fails to match; if it \emph{does} match, the value of the test \texttt{(> a b)} is returned.  As before, if the clause is true, the function is called recursively.

\medskip
In CL's Alexandria library, there is a \texttt{destructuring-case}.  In EL, dash has \texttt{-when-let} (as a clause) and \texttt{-let} can do destructuring.
\end{frame}

\begin{frame}[fragile]{Pattern Matching and Guards}
  Here is a cleaner way of handling the problem:
\begin{verbatim}
(defun func (sum val)
  (destructuring-bind (a . b) val
    (+ sum (if (> a b) a b))))
func
(reduce #'func
  '((3 . 1) (4 . 1) (2 . 3) (-1 . 2))
  :initial-value 0)
12
\end{verbatim}
\medskip
Incidentially, \texttt{loop} can also destructure:
\begin{verbatim}
(loop for (a . b) in '((3 . 1) (4 . 1) (2 . 3) (-1 . 2))
      summing (max a b))
12
\end{verbatim}

\end{frame}
\end{document}
