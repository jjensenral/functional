% Created 2022-01-16 Sun 11:18
% Intended LaTeX compiler: pdflatex
\documentclass[presentation]{beamer}
\usepackage[utf8]{inputenc}
\usepackage[T1]{fontenc}
\usepackage{graphicx}
\usepackage{grffile}
\usepackage{longtable}
\usepackage{wrapfig}
\usepackage{rotating}
\usepackage[normalem]{ulem}
\usepackage{amsmath}
\usepackage{textcomp}
\usepackage{amssymb}
\usepackage{capt-of}
\usepackage{hyperref}
\usepackage{xypic}
\usepackage{tabularx}
\xyoption{pdf}
\usetheme{default}
\newcommand\cat[1]{\ensuremath{\mathcal{#1}}}
\newcommand\hm[2]{\hom(#1,#2)}
\newcommand\nat[2]{{#1}\overset{\cdot}{\rightarrow}{#2}}
\newcommand\tup[1]{\langle #1\rangle}
\newcommand\id[1]{\mathrm{id}_{#1}}
\author{jens.jensen@stfc.ac.uk \\0000-0003-4714-184X\\CC-BY 4.0}
\date{April 24, 2022}
\title{Functional Programming 4}
\hypersetup{
 pdfauthor={jens.jensen@stfc.ac.uk},
 pdftitle={Functional Programming 4},
 pdfkeywords={functional, monad, programming theory},
 pdfsubject={functional programming},
 pdfcreator={Emacs 27.1 (Org mode 9.3) then by hand}, 
 pdflang={English}}
\begin{document}

\maketitle
\begin{frame}{Outline of Talks}

  \begin{itemize}
    \item Previous talks (talks 1-3):
    \begin{itemize}
\item Introibo
\item Pure Functional Programming Principles
    \item Mapping
    \item Labels and naming
    \item Lists
    \end{itemize}
  \item This talk (Talk 4):
    \begin{itemize}
    \item Advanced(ish) Topics
    \end{itemize}
\item Impure Functional? Side Effects
\item Category Theory
\item Categories and Functions
\item Categories and Computation

  \end{itemize}
Still written in the author's spare time!

\medskip

Very much a personal perspective, and not following any particular textbook.  Using \emph{meditations} and \emph{exercises} -- solutions to all exercises given during the talks.

\end{frame}


\section{Introibo}
\label{sec:org0edb596}


\begin{frame}{Superpowertools -- Lemonodor fame awaits!}
  Summarising the Lisp functional programming Super Power Power Tools we have met (plus one from an earlier talk):
  \begin{itemize}
  \item \emph{Recursion}, the elegant engine of functional programming;
  \item \texttt{list} and list tools, designed for the first/rest paradigm;
  \item \texttt{cond} and friends for dividing and conquering;
  \item Mapping functions (and \texttt{reduce}) implement functional patterns concisely (including recursion, without recursing)
  \item Sequence functions such as \texttt{reduce}
  \item Functions: \texttt{funcall}, \texttt{lambda}, \texttt{apply}; multivalued in CL
  \item \texttt{lambda} lists in Lisp are much more powerful than other functional languages -- more powerful than other languages;
  \item \texttt{flet} and \texttt{labels} for defining local functions;
  \item \emph{macros}, because Lisp macros are superpowertools for everything
  \item \emph{stubs}, and \texttt{trace} (the latter CL only, but see later)
  \end{itemize}
\end{frame}

\section{Advanced(ish) Functional Programming}
\begin{frame}{Common/Advanced(ish) Features of Functional Languages}
\label{sec:org95ddb74}

\begin{enumerate}
\item Lambda (anonymous (unnamed) functions) and \emph{currying}
\item List comprehension
\item Functions -- mutually recursive, higher order
\item Symbols
\item Tail recursion
\item Scope and extent (Lisp)
\item Types and type inference
\item Branch-on-pattern-matching and guards
\item Memoisation
\item Lazy evaluation types
\item Pipes (not the lazy kind) style composition
\begin{itemize}
\item  \(h(g(f(x)))\equiv\texttt{(h (g (f x)))}\equiv{}x\vert{f}\vert{g}\vert{h}\)
\end{itemize}
\item Monads: theoretical framework for types and computation
\item Applied monads: Maybe, Arrays
\item Bonus section for survivors of MonadLand: Lisp Hacking
\end{enumerate}

\end{frame}

\begin{frame}[fragile]{Common Features of Functional Languages - Currying}
\begin{itemize}
\item Currying (binding some but not all function arguments)
\begin{itemize}
\item In this example \texttt{+} is curryed, binding its second argument to 2
\begin{itemize}
\item (Assuming \texttt{+} is considered as a function of \emph{two} args (which it isn't))
\end{itemize}
\item ``Proper'' functional languages like F\# do this more elegantly
\begin{itemize}
\item As long as we bind the \emph{first argument}
\item E.g.\ \texttt{+ 2} would be the function that adds two to something
\end{itemize}
\end{itemize}
\end{itemize}

\begin{verbatim}
(mapcar (lambda (x) (+ x 2)) '(1 2 3 4))
(3 4 5 6)
\end{verbatim}

\begin{itemize}
\item Hence the \emph{type} notation of a function \(+::\mathrm{int}\rightarrow\mathrm{int}\rightarrow\mathrm{int}\)
\item After currying the (first) argument, the expr has type \(::\mathrm{int}\rightarrow\mathrm{int}\)
\item Of course in Lisp, \texttt{\#'+} takes any number of arguments (incl 0)
\item In talk 2 we had \texttt{(apply \#'mapcar \#'list rows)} where \texttt{apply} effectively curryes \texttt{mapcar} by binding \texttt{\#'list} as its first argument
\end{itemize}
\end{frame}

\begin{frame}[fragile]{Lambdas}
  Almost but not quite a translation of $\lambda$ calculus.  Taking the Quine
  \begin{equation*}
    ((\lambda x. (x x)) (\lambda x. (x x)))
  \end{equation*}
  becomes in EL
\begin{verbatim}
((lambda (x) (list x x)) (lambda (x) (list x x)))
((closure (t) (x) (list x x)) (closure (t) (x) (list x x)))
\end{verbatim}
which is a Quine if you appreciate that evaluating \texttt{lambda} \texttt{coerce}s the \texttt{lambda} expression into a function
\begin{verbatim}
(lambda (v) (+ v 2))
(closure (t) (v) (+ v 2))
(coerce (lambda (v) (+ v 2)) 'function)
(closure (t) (v) (+ v 2))
\end{verbatim}
\end{frame}

\begin{frame}[fragile]{Lambdas}
  In CL, functions can return multiple values (including none) but special tools are needed to capture the not-first values:
\begin{verbatim}
(floor 12 7)
1
5
(multiple-value-bind (q r) (floor 12 7)
  (+ (* q 7) r))
12
\end{verbatim}
When not used in a multiple value context, only \texttt{floor}'s first value is used (as in EL where it returns only one value):
\begin{verbatim}
(+ (floor 12 7) 3)
4
\end{verbatim}
\end{frame}
\begin{frame}[fragile]{Lambdas}
Both \texttt{lambda} and \texttt{let} create variable bindings:
\begin{verbatim}
((lambda (v)
  (+ v (let ((v 2)) (* 3 v))))
  6)
12
\end{verbatim}
\begin{itemize}
\item The inner \texttt{v} is bound by \texttt{let} to the value 2
\item The outer \texttt{v} is bound by \texttt{lambda} to the value 6
\item It's the same symbol \texttt{v} but the inner binding shadows the outer during execution of the \texttt{let}
\item The outermost parenthesis 
\end{itemize}
\end{frame}

\begin{frame}[fragile]{List comprehension}
\emph{List comprehension} is the idea of constructing elements
$$(f(x,y,\ldots)\vert x\in A, y\in B(x),\ldots)$$
analogous to how sets are constructed (this is obviously not a precise definition but a it-will-do-for-now expression of the idea).

\medskip
Arguably, list comprehension for Lisp is \texttt{loop}, even though it looks imperative sometimes:
\begin{verbatim}
(loop for i from 1 to 10 collecting (* i i))
(1 4 9 16 25 36 49 64 81 100)
\end{verbatim}
but this could be done with \texttt{iota} functionally.  It can express actions on lists (and vectors) equally naturally:
\begin{verbatim}
(loop for a in '(1 2) nconcing
  (loop for b in '(3 4) collecting (cons a b)))
((1 . 3) (1 . 4) (2 . 3) (2 . 4))
\end{verbatim}
\end{frame}

\begin{frame}[fragile]{List comprehension}
  Another common case is calling a function (with side effects!) $n$ times, collecting the results (which means we can't just use \texttt{dotimes}):
\begin{verbatim}
(loop repeat 10 collect (random 10))
(4 9 1 2 9 3 8 0 0 8)
\end{verbatim}
The alternative could be
\begin{verbatim}
(map 'vector (lambda (x) (random 10)) (make-vector 10 0))
[8 3 3 9 8 6 5 0 5 3]
\end{verbatim}
which would have made more sense with CL's \texttt{map-into}
\begin{verbatim}
(let ((v (make-array 10)))
  (map-into v
    (lambda (x) (declare (ignore x)) (random 10))
    v))
\end{verbatim}
\end{frame}

\begin{frame}{Mutually recursive functions}
  A puzzle: given a number $N$, create an array of length $2N$ with each of the digits $1,\ldots,N$ occurring precisely twice, such that for every digit $k\in\{1,\ldots,N\}$ there is $k$ digits between the two occurrences of the digit $k$.

  \medskip
  Example for $N=3$: \texttt{312132}.  Starting digits may be given, e.g.\ starting with \texttt{*1*1**}

  \medskip
  Solving it functionally, the original design had three functions calling each other -- now we're down to two...
\end{frame}


\begin{frame}{Functions -- Mutually recursive functions}
  \resizebox{!}{8cm}
            {
              \includegraphics{mutual.pdf}
            }
\end{frame}

\begin{frame}[fragile]{Functions -- Hooks and callbacks}
Getting one value out of a deeply nested computation: throw it.
\begin{verbatim}
(... (when (goalp state) (throw 'found state)) ...)
\end{verbatim}
(for some hypothetical function \texttt{goalp}.)  When we need to collect multiple values, a callback makes sense:
\begin{verbatim}
(... (when (goalp state) (funcall cb state)) ...)
\end{verbatim}
Later we shall look at other ways of solving the same problem.
\end{frame}

\begin{frame}[fragile]{Functions -- Hooks and callbacks}
  Suppose for a moment we want Fibonacci numbers (to which we will also return later), starting with the CL version:
\begin{verbatim}
(defun fib (k &optional (a 0) (b 1))
  (if (zerop k) b
    (fib (1- k) b (+ a b))))
FIB
(fib 10)
89
\end{verbatim}
\end{frame}

\begin{frame}[fragile]{Functions -- Hooks and callbacks}
  In CL we can now do
\begin{verbatim}
(trace fib)
(FIB)
(fib 5)
  0: (FIB 5)
    1: (FIB 4 1 1)
      2: (FIB 3 1 2)
        3: (FIB 2 2 3)
          4: (FIB 1 3 5)
            5: (FIB 0 5 8)
            5: FIB returned 8
          4: FIB returned 8
        3: FIB returned 8
      2: FIB returned 8
    1: FIB returned 8
  0: FIB returned 8
8
(untrace fib)
T
\end{verbatim}
\end{frame}

\begin{frame}[fragile]{Functions -- Hooks and callbacks}
In EL, the \texttt{fib} function could look like this:
\begin{verbatim}
(defun fib (k &optional a b)
  (if (zerop k) (or b 1)
    (fib (1- k) (or b 1) (+ (or a 0) (or b 1)))))
fib
(fib 10)
89
\end{verbatim}

\end{frame}

\begin{frame}[fragile]{Functions -- Hooks and callbacks}
  EL uniquely allows us to alter functions with \emph{advice}:
\begin{verbatim}
(defun my-trace (name func args)
  ;; Enter function
  (princ (format "%s: %s\n" name args))
  ;; Call function and remember value
  (let ((val (apply func args)))
    ;; Print result
    (princ (format "%s => %s\n" name val))
    val))
(advice-add 'fib :around
  (lambda (func &rest args) (my-trace 'fib func args))
  '((name . trace)))
nil
\end{verbatim}
In this respect, advice work quite a lot like methods (specifically, standard method combination) in CLOS
\end{frame}
\begin{frame}[fragile]{Functions -- Hooks and callbacks}
\begin{verbatim}
(fib 5)
fib: (5)
fib: (4 1 1)
fib: (3 1 2)
fib: (2 2 3)
fib: (1 3 5)
fib: (0 5 8)
fib => 8
fib => 8
fib => 8
fib => 8
fib => 8
fib => 8
8
(advice-remove 'fib 'trace)
nil
\end{verbatim}
\end{frame}

\begin{frame}[fragile]{Functions -- Hooks and callbacks}
Emacs uses \emph{hooks} a lot, callbacks called at specific times.  Normal hooks are single function or (more or less) lists of functions to be called without arguments:
\begin{verbatim}
text-mode-hook
(text-mode-hook-identify)
(add-hook 'text-mode-hook 'auto-fill-mode)
(auto-fill-mode text-mode-hook-identify)
c++-mode-hook
(irony-mode)
(add-hook 'c++-mode-hook
  (lambda () (message "Hack and be merry!"))
  10)
(irony-mode (closure (t) nil (message "Hack and be merry!")))
\end{verbatim}
\end{frame}

\begin{frame}[fragile]{Functions -- Higher order functions}
\begin{verbatim}
(defun compose (f g)
  "Compose two functions calling g first then f"
  (lambda (&rest data) (funcall f (apply g data))))
compose
(funcall (compose #'sqrt #'+) 2 3 4)
3.0
(mapcar (compose #'sqrt #'abs) '(-1 -4 9 -16))
(1.0 2.0 3.0 4.0)
\end{verbatim}
\end{frame}

\begin{frame}[fragile]{Functions -- Higher order functions}
The CL standard (ANSI X3J13) has tried to deprecate the \texttt{-if-not} functions in favour of the \texttt{-if}s:
\begin{verbatim}
(remove-if-not #'evenp '(1 2 3 4 5 6))
(2 4 6)
(remove-if (complement #'evenp) '(1 2 3 4 5 6))
(2 4 6)
\end{verbatim}
The first of these works fine in EL but EL does not have \texttt{complement}.  However, we can easily write it:
\begin{verbatim}
(defun complement (func)
  "Logical complement of a function of any number of args"
  (lambda (&rest args)
    (not (apply func args))))
\end{verbatim}
Notice that no \texttt{funcall} is required when we use it (why not?)
\end{frame}

\begin{frame}[fragile]{Symbols}
  \emph{Symbols} are one of the important tools of Lisp, as
  \begin{itemize}
  \item Names -- of variables, functions, catches, types, structures, classes, etc.
  \item Enums (eg.\ \texttt{(list 'Jan 'Feb 'Mar ...)})
  \item Keys for hooks or looking up stuff
  \item Anywhere else where you need an atom
  \end{itemize}

What is going on here (this is EL but could be CL with one difference):
\begin{verbatim}
(atom nil)
t
(eq 'fred 'Fred)
nil
(eq 'fred 'fred)
t
(eq (make-symbol "fred") (make-symbol "fred"))
nil
\end{verbatim}

\end{frame}

\begin{frame}[fragile]{Symbols}
  A symbol has precisely three things:
  \begin{itemize}
  \item A \emph{name} (or more accurately, a \emph{print name}):
  \end{itemize}
\begin{verbatim}
(symbol-name 'fred)
"fred"
(symbol-name (make-symbol "wilma"))
"wilma"
\end{verbatim}
\begin{itemize}
\item A \emph{package} (sort of, EL doesn't really do packages)
\item A \emph{property list}:
\end{itemize}
\begin{verbatim}
(symbol-plist 'fred)
nil
(symbol-plist 'car)
(byte-compile byte-compile-one-arg byte-opcode byte-car ...
... byte-optimizer byte-optimize-predicate
    side-effect-free t)
\end{verbatim}
\end{frame}

\begin{frame}[fragile]{Symbols}
If we take the plist first, it provides key/value lookup.  Each symbol has its own plist.
\begin{verbatim}
(get 'fred 'foo)
nil
(setf (get 'fred 'foo) 'blemps)
blemps
(get 'fred 'foo)
blemps
(setf (get 'fred 'bar) 17)
17
(symbol-plist 'fred)
(foo blemps bar 17)
\end{verbatim}
It works exactly like an alist except
\begin{itemize}
\item for hysterical raisins the structure is different
\item it returns the value rather than the key/value
\item it does not normally shadow
\item it has fewer lookup functions (no equiv of \texttt{assoc-if})
\end{itemize}
\end{frame}

\begin{frame}[fragile]{Symbols}
Unlike CL, Emacs uses property lists extensively:
\begin{verbatim}
(get 'sort 'side-effect-free)
nil
(get 'mapcar 'side-effect-free)
nil
(get 'cons 'side-effect-free)
error-free
(get '+ 'side-effect-free)
t
\end{verbatim}
Keys should normally be symbols, though values need not be
\begin{verbatim}
(setf (get 'fred "name") "jones")
"jones"
(get 'fred "name")
nil
\end{verbatim}
\end{frame}

\begin{frame}[fragile]{Symbols}
Removing a key (``tag'') from a plist is slightly different:
\begin{verbatim}
(remprop 'fred 'foo)
t
(symbol-plist 'fred)
(bar 17)
\end{verbatim}

Altering plists of EL's built in functions is not advisable!

\medskip
There are functions to work directly on plists:
\begin{verbatim}
(let ((plist (list 'a 1 'b 2)))
  (setf (getf plist 'c) 3)
  (remf plist 'b)
  plist)
(c 3 a 1)
\end{verbatim}
\end{frame}

\begin{frame}[fragile]{Symbols}
Let's return to the packages.  In simplified terms, a package is a context where symbols can be looked up; when the Reader first reads a symbol it is created:
\begin{verbatim}
(eq 'fred 'fred)
t
\end{verbatim}
The same symbol is referenced twice, so obviously \texttt{eq}.  However, a small number of functions and macros create symbols that are \emph{not in any package}, so cannot be looked up again:
\begin{verbatim}
(make-symbol "fred")
fred
(eq (make-symbol "fred") 'fred)
nil
\end{verbatim}
Here \texttt{'fred} is in the package but \texttt{make-symbol} creates one that isn't, so they can never be \texttt{eq}: effectively, symbols not in the (same) package cannot be \texttt{eq} even if they have the same name.
\end{frame}

\begin{frame}[fragile]{Symbols}
We use this construction to write macros (this is CL from the author's Advent of Code 14/12/2021):
\begin{verbatim}
(defmacro incf-list2-entry (list x y delta)
  "Create or add delta to the entry in list for the pair x y"
  (let ((bzz (gensym)) (garg (gensym)))
    `(let* ((,bzz (cons ,x ,y))
            (,garg (assoc ,bzz ,list :test #'equal)))
       (if ,garg
           (incf (cdr ,garg) ,delta)
           (setf ,list (acons ,bzz ,delta ,list)))
       ,list)))
\end{verbatim}
(For EL code, leave out the \texttt{:test \#'equal}).  Dodgy names apart, the macro is a variation on \texttt{incf} for alists of the form
\begin{verbatim}
(((C . B) . 1) ((N . C) . 1) ((N . N) . 1))
\end{verbatim}
-- looking up a pair \texttt{(x . y)}, it creates or adds to the pair.
\end{frame}

\begin{frame}[fragile]{Symbols}
The Reader can also create symbols not-in-any-package using Reader macros:
\begin{verbatim}
(eq '#:fred 'fred)
nil
(eq '#:fred '#:fred)
nil
(symbol-name '#:fred)
"fred"
\end{verbatim}
The Reader macro \texttt{\#:} creates a symbol with the same print name but it is in no package; the next time the Reader reads the same characters, a different symbol is created.
\end{frame}

\begin{frame}[fragile]{Symbols}
Since it is created by the Reader (before Eval), we need to use other Reader constructions to reference the symbol (EL code):
\begin{verbatim}
(defmacro incf-list2-entry (list x y delta)
  `(let* ((#1=#:bzz (cons ,x ,y))
          (#2=#:garg (assoc #1# ,list)))
       (if #2#
           (incf (cdr #2#) ,delta)
           (setf ,list (acons #1# ,delta ,list)))
       ,list))
incf-list2-entry
(let ((mylist (list '((a . b) . 3))))
  (incf-list2-entry mylist 'a 'b 2))
(((a . b) . 5))
(let ((mylist (list '((a . b) . 3))))
  (incf-list2-entry mylist 'x 'y 17))
(((x . y) . 17) ((a . b) . 3))
\end{verbatim}
\end{frame}

\begin{frame}[fragile]{Symbols}
Now let's return to symbols that \emph{are} in ``the'' package, called \emph{interned} symbols (symbols not in a package are called \emph{uninterned}):
\begin{verbatim}
(eq 'fred (intern "fred"))
t
(eq (intern "fred") (intern "fred"))
t
\end{verbatim}

\medskip
Symbols are atoms and not sequences; if we really wanted to turn a list of two elements \texttt{(fizz buzz)} into a single symbol \texttt{fizzbuzz}, we'd have to do (EL code)
\begin{verbatim}
(defun join-symbols (w)
  (if (endp (cdr w)) ; 0-1 elt
      (first w)
    (intern (concat (symbol-name (first w))
                    (symbol-name (second w))))))
join-symbols
(join-symbols '(fizz buzz))
fizzbuzz
\end{verbatim}
\end{frame}

\begin{frame}[fragile]{Symbols}
Symbols -- whether interned or not -- have few restrictions on their names: this is a single symbol:
\begin{verbatim}
(symbolp 'b^2-4*a*c/2*a)
t
\end{verbatim}
but notice we cannot add parentheses for the denominator as the Reader would start constructing a list.  Likewise, \texttt{1-} is a symbol, the name of the function
\begin{verbatim}
(symbol-function #'1-)
#<subr 1->
\end{verbatim}
Also note that in CL (but not in EL), the Reader maps all symbols to upper case:
\begin{verbatim}
* (eq 'fred 'FRED)
T
* 'fred
FRED
\end{verbatim}
\end{frame}

\end{document}
