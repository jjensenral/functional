% Created 2022-01-16 Sun 11:18
% Intended LaTeX compiler: pdflatex
\documentclass[presentation]{beamer}
\usepackage[utf8]{inputenc}
\usepackage[T1]{fontenc}
\usepackage{graphicx}
\usepackage{grffile}
\usepackage{longtable}
\usepackage{wrapfig}
\usepackage{rotating}
\usepackage[normalem]{ulem}
\usepackage{amsmath}
\usepackage{textcomp}
\usepackage{amssymb}
\usepackage{capt-of}
\usepackage{hyperref}
\usepackage{xypic}
\usetheme{default}
\newcommand\cat[1]{\mathcal{#1}}
\newcommand\hm[2]{\hom(#1,#2)}
\newcommand\nat[2]{{#1}\overset{\bullet}{\rightarrow}{#2}}
\author{jens.jensen@stfc.ac.uk \\0000-0003-4714-184X\\CC-BY 4.0}
\date{January 30, 2022}
\title{Functional Programming}
\hypersetup{
 pdfauthor={jens.jensen@stfc.ac.uk},
 pdftitle={Functional Programming},
 pdfkeywords={functional, monad, programming theory},
 pdfsubject={functional programming},
 pdfcreator={Emacs 27.1 (Org mode 9.3) then by hand}, 
 pdflang={English}}
\begin{document}

\maketitle
\begin{frame}{Outline}

  \begin{itemize}
    \item This talk
    \begin{itemize}
\item Introibo
\item Pure Functional Programming Principles
    \end{itemize}
  \item Future talks
    \begin{itemize}
  \item Advanced(ish) Functional Programming
\item Impure Functional? Side Effects
\item Category Theory
\item Categories and Functions
\item Categories and Computation
    \end{itemize}
\end{itemize}

\end{frame}




\begin{frame}
  Let us get together and study.  As the Poet once said,
  \begin{quote}
The full-spread power of Emacs is calming and excellent to the soul
  \end{quote}
The same is true for category theory, and Lisp.  And mostly true for functional programming.
\end{frame}

\begin{frame}{This talk}
  This talk is the first of $n>2$ on functional programming with (E)Lisp; it was given to RSEmacs
  01 Feb 2022.

   \url{https://m-x-research.github.io/}
\end{frame}

\begin{frame}{Rules and Terminology}
  \begin{itemize}
  \item Interrupt to ask questions -- we will be covering a lot of different ground.  Best to ask before we are on a different topic.  The first part of this talk is meant to be more interactive, the second (``monadland'') is less so.
  \item Some terminology:
    \begin{itemize}
    \item ``s.t.'' == ``such that''
    \item iff == ``if and only if''
    \item I might use the word ``map'' for $\mapsto$, as in \texttt{sqrt} maps $9$ to $3$
    \item EL == Emacs Lisp; CL = Common Lisp (\texttt{M-x slime})
    \item Inferior languages = languages which are not Lisp
    \end{itemize}
  \item MonadLand Arrows (to be explained later):
    \begin{itemize}
    \item $\rightarrow$ for arrow, function, functor 
    \item $\nat{}{}$ for natural transformation
    \item $\rightharpoonup$ for adjunction
    \end{itemize}
  \end{itemize}
\end{frame}

\begin{frame}{Functional programming}
\label{sec:org033efd0}

Introduction to Functional Programming (Theory and Practice) with (Emacs) Lisp (one tends to use parentheses a lot (like this one) when writing Lisp talks)

\smallskip
The author will attempt a synthesis of four different views:

\begin{itemize}
\item CS (IANACS): types, lambda calculus
\item Maths: category theory
\item "Programming theory": pure and impure functional
\item "Programming practice": Emacs Lisp (instead of, say, Haskell)
\end{itemize}

Written in the author's Copious Spare Time™, any opinions (and mistakes) are his own (unless stated otherwise).  Lisp examples are ELisp unless indicated otherwise.  It assumes familiarity with ELisp.

\end{frame}

\begin{frame}{Principles of Functional Programming}
\label{sec:orgf4f5447}

\emph{Pure} functional programs are ``defined'' by these basic features:

\begin{itemize}
\item Variables (\texttt{let}, \texttt{let*} etc) are immutable
  \begin{itemize}
  \item No \texttt{setq}/\texttt{setf} or modifier ...\texttt{f} (eg \texttt{incf})
  \item Use \emph{recursion} instead of \texttt{loop} or \texttt{do}, \texttt{do*}, etc
  \end{itemize}
\item Functions are called \emph{without side effects}
  \begin{itemize}
  \item No \texttt{n}... functions (\texttt{nreverse} etc) ...
  \item ... except when the side effect has no effect (see later)
  \end{itemize}
\item Can pass functions as argument
\item Divide and conquer strategy for solving problems
\end{itemize}

Later we shall zoom in on advanced features of functional programming in general and in Lisp(s) in particular.

\end{frame}

\begin{frame}{Why Functional?}
  \begin{itemize}
  \item Elegance (much of the time)
  \item Correctness
    \begin{itemize}
    \item Programmer-readable correctness (loop invariants become parameter constraints)
    \item Compiler-inferred correctness (stronger type checks)
    \end{itemize}
  \item Cleaner code (examples later)
  \item Like learning Lisp, it makes you a Better Programmer\copyright
  \end{itemize}
\end{frame}

\begin{frame}{Why not functional?}
We shall see examples of these during the talk (and how to mitigate (some of) them):
  \begin{itemize}
  \item Inefficiencies:
    \begin{itemize}
    \item copy lists/arrays to update them
    \item temporary lists being created
    \item more function calls? (stack frames etc)
      \begin{itemize}
      \item \texttt{setq} is usually cheaper than a function call
      \end{itemize}
    \end{itemize}
  \item Some patterns are difficult to program functionally
    \begin{itemize}
    \item Arrays are a good example, particularly multidimensional
      \begin{itemize}
      \item Particularly arbitrary-dimensional (generic) arrays which most languages other than CL struggle with
      \end{itemize}
    \item In fact, arrays have theoretical, practical and lispological ``issues''
    \end{itemize}
  \end{itemize}
\end{frame}

\begin{frame}[fragile]{Functions as arguments}
\begin{verbatim}
((lambda (x y) (+ x y)) 2 3)
5
(funcall (lambda (x y) (+ x y)) 2 3)
5
(let ((plus #'+)) (funcall plus 2 3))
5
(defun fred (x)
  (funcall (if (evenp x)
	       #'sqrt
	     (lambda (y) (- y)))
	   x))
fred
(mapcar #'fred '(1 2 3 4 5 6))
(-1 1.4142135623730951 -3 2.0 -5 2.449489742783178)
\end{verbatim}
\end{frame}

\begin{frame}[fragile]{Example}
\label{sec:org25d9a35}

\begin{verbatim}
(defun fact (k)
  (if (zerop k) 1 (* k (fact (1- k)))))
fact
(fact 12)
479001600
\end{verbatim}
\begin{itemize}
\item Uses \(n!=n(n-1)!\) for \(n>0\), and \(0!=1\)
\item Should check for \(n<0\) or \(n\) not a number
\begin{itemize}
\item In "true" functional languages like Haskell and F\#, the type is \emph{inferred} (from \texttt{*} and \texttt{-})
\item (Some) Lisps do something similar but usually less strict
\end{itemize}
\end{itemize}

\end{frame}
\begin{frame}[fragile]{Example - A Functional Koan}
  This is CL from the author's Advent of Code(AoC) 2021, 01 Dec.
\begin{verbatim}
(defun solve1 (data)
  "Count number of numbers that increase in a list"
  (count-if #'plusp (mapcar #'- (cdr data) data)))
\end{verbatim}
\begin{itemize}
\item \texttt{mapcar} maps as long as the shortest of its input lists
\item In efficiency terms, an intermediate list of length $n-1$ (for input of length $n$) is created
  \begin{itemize}
  \item This list will need to be \emph{gc}ed, eventually
  \item As a (functional) Lisp programmer, you have to consider efficiency - both memory and time
  \end{itemize}
\item EL's \texttt{mapcar} only accepts unary functions (and a single list)
\end{itemize}
\end{frame}
\begin{frame}[fragile]{Example - A Less Elegant Koan}
Another short example from the same code base:
\begin{verbatim}
(defun sum-3-window (list)
  "For a list of inputs, return the sum of
   sliding length 3 windows"
  (mapcar (lambda (a b c) (+ a b c))
          list (cdr list) (cddr list)))
\end{verbatim}
\begin{itemize}
\item No unnecessary lists created
\item Exercise: what if the length were passed in?
\item Again won't work with EL's \texttt{mapcar}
\end{itemize}
\end{frame}
\begin{frame}[fragile]{Example -- A Solver Wrapper}
Also from the author's AoC21, allowing different inputs for testing:
\begin{verbatim}
(defun solve (data solve-func)
  (cond
    ((streamp data) (funcall solve-func data))
    ((pathnamep data)
      (with-open-file (foo data
          :direction :input :if-does-not-exist :error)
        (solve foo solve-func)))
    ((stringp data)
      (solve (make-string-input-stream data) solve-func))
    (t (error "Cannot process ~A" (type-of data)))))
\end{verbatim}
\begin{itemize}
\item This is CL rather than EL but you can see how it works
\item In CL, pathnames are distinct from strings (though the latter can implicitly convert to the former)
\item \texttt{etypecase} would have been a better choice for this particular problem
\end{itemize}

\end{frame}

\begin{frame}{Example: an $n$-ary plus}
Let's assume we have a $2$-ary plus (\texttt{\#'+}) and we want to build an $n$-ary plus.

\medskip
Mathematically(ish), we define:
\begin{alignat*}{2}
  \mathrm{plus}() &= 0 && \qquad \text{(empty list);} \\
  \mathrm{plus}(a) & = a && \qquad \text{(one element);} \\
  \mathrm{plus}(a,b \cdots) &= a+\, \mathrm{plus}(b \cdots) && \qquad \text{(recursively)}
\end{alignat*}
The effect is that of
$$\mathrm{plus}(a,b,c,d)=a+(b+(c+d))$$
\end{frame}
\begin{frame}[fragile]{$n$-ary \texttt{+}}
Turning this into code, we get
\begin{verbatim}
(defun my+ (my-list)
  "Sum an arbitrary number of elements"
  (cond
    ((= (length my-list) 0) 0)
    ((= (length my-list) 1) (first my-list))
    (t (+ (first my-list) (my+ (rest my-list))))))
\end{verbatim}

\medskip
This works and is functional, but it is not a good implementation for several reasons.  Your meditation exercise (before turning to the next slide) is to ponder why.
\end{frame}

\begin{frame}{$n$-ary \texttt{+}}
  \begin{itemize}
  \item (Usability) First of all, we would like to call \texttt{(my+ 1 2 3 4)} instead of \texttt{(my+ '(1 2 3 4))}
  \item (Efficiency) Second, the \texttt{length} function is called which will calculate the length every time; we just need to know if the list is empty (or has a single element)
  \item (Correctness) Calling the function on a long list will exhaust the stack (which could lead to heisenbugs as we don't know how much space is left on the stack)
  \item (Correctness) floatologically speaking, adding numbers na\"\i vely can lead to precision loss etc.\ (we shall ignore this in this talk)
  \end{itemize}

\end{frame}

\begin{frame}[fragile]{$n$-ary \texttt{+}}
\begin{verbatim}
(defun my+ (&rest my-list)
  "Sum an arbitrary number of elements"
  (cond
    ((endp my-list) 0)
    ((endp (cdr my-list)) (first my-list))
    (t (+ (first my-list) (apply #'my+ (rest my-list))))))
\end{verbatim}
Notice the \emph{four} differences with the earlier version.  (\texttt{car} is synonymous with \texttt{first} and \texttt{cdr} with \texttt{rest}; the choice is mostly stylistic.  The author prefers \texttt{first}/\texttt{rest} for functional divide-and-conquer and \texttt{c*r} for everything else (see also Naming, later))

\smallskip
Meditation: there is a more subtle performance issue (hint: how many times is \texttt{length}, resp.\ \texttt{endp} called for a list of length $n\geq 2$?)  And we haven't dealt (yet) with the \emph{correctness} issues.
\end{frame}
\begin{frame}[fragile]{$n$-ary \texttt{+}}
Of course \texttt{endp} is much, much cheaper than \texttt{length} in general
\begin{verbatim}
(defun my+ (&rest my-list)
  "Sum an arbitrary number of elements"
  (if (endp my-list) 0 (my+1 my-list)))

(defun my+1 (my-list)
  "Helper function for my+: sum elements of a non-empty list"
  (if (endp (cdr my-list)) (first my-list)
    (+ (first my-list) (my+1 (rest my-list)))))
\end{verbatim}
Performance testing would be needed to decide whether it is worth it

\smallskip
Meditation exercise: Could we have passed in the function as well (\`a la \texttt{mapcar})?  Also, we still haven't dealt with the \emph{correctness} issues.
\end{frame}

\begin{frame}[fragile]{$n$-ary \texttt{+}}
   Of course, Emacs's \texttt{+} is already $n$-ary.  Using built-ins:
\begin{verbatim}
(+ 1 2 3 4 5)
15
(apply #'+ '(1 2 3 4 5))
15
(funcall #'+ 1 2 3 4 5)
15
(reduce #'+ '(1 2 3 4 5))
15
\end{verbatim}
or with the function bound in a variable:
\begin{verbatim}
(let ((plus #'+)) (funcall plus 3 4))
7
(let ((plus #'+)) (apply plus '(3 4)))
7
\end{verbatim}

Meditation: why do \texttt{reduce} and \texttt{apply} seem to do exactly the same thing?

\end{frame}
\begin{frame}[fragile]{Elegance vs Efficiency}
In a list \texttt{seqs} of sequences (again from AoC), all have the same length:
\begin{verbatim}
(apply #'= (mapcar #'length seqs))
\end{verbatim}
\medskip

Meditation: this is short, functional and elegant.

\smallskip
Usually, ``short'' and ``functional'' imply ``elegant'' (and \emph{inefficient}...?)

\medskip
It is indeed not the most \emph{efficient} implementation in general.  Why not?  Does it matter?  If it matters, how can it fixed (functionally)?
\end{frame}

\begin{frame}[fragile]{Example -- legitimate use of side effects}
  Sometimes, using functions with side effects (i.e.\ they alter their argument) is safe, correct and functional:
\begin{verbatim}
(defun integer-digits (n radix)
  "Split a non-negative integer into its digits"
  (labels ((i-dig (n)
             (if (< n radix) (list n)
               (let ((q (floor n radix))
                     (r (mod n radix)))
                   (cons r (i-dig q))))))
    (nreverse (i-dig n))))
(integer-digits 2345 10)
(2 3 4 5)
\end{verbatim}
(comments on next slide)
\end{frame}
\begin{frame}
\begin{itemize}
\item \texttt{labels} define local functions that can call themselves (or each other)
\item The temporary list is an r-value (in C++-speak), so using \texttt{nreverse} on it avoids creating more cons cells
  \begin{itemize}
  \item Indeed, it is illegal (or at least a very bad idea) to call \texttt{nreverse} \emph{et al} on l-values
  \item Not all \texttt{n} functions are consistently named: \texttt{sort} and \texttt{stable-sort} also destroy their arguments
  \end{itemize}
\item In CL, \texttt{floor} returns both the quotient and the remainder, so is more efficient
\end{itemize}

\end{frame}

\begin{frame}[fragile]{Example -- Reduce Koan}
  The inverse of \texttt{integer-digits}:
\begin{verbatim}
(defun digits-to-integer (digits radix)
  "Turn list of digits into integer with given radix"
  (reduce (lambda (a b) (+ (* a radix) b)) digits))
(digits-to-integer '(2 3 4 5) 10)
2345
\end{verbatim}
Some items for when you meditate on this code:
\begin{itemize}
\item Can we express the \emph{type constraint} that \texttt{digits} is a list of integers $k$ s.t.\ $0\leq k<$\,\texttt{radix}?
  \begin{itemize}
  \item Actually it just has to be a \emph{sequence}
  \item \texttt{(digits-to-integer "ABCD" 100)} $=$ \texttt{65666768}
  \end{itemize}
\item What happens if \texttt{digits} is of length 1?  Of length 0?
\end{itemize}

\end{frame}
\end{document}
