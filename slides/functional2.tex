% Created 2022-01-16 Sun 11:18
% Intended LaTeX compiler: pdflatex
\documentclass[presentation]{beamer}
\usepackage[utf8]{inputenc}
\usepackage[T1]{fontenc}
\usepackage{graphicx}
\usepackage{grffile}
\usepackage{longtable}
\usepackage{wrapfig}
\usepackage{rotating}
\usepackage[normalem]{ulem}
\usepackage{amsmath}
\usepackage{textcomp}
\usepackage{amssymb}
\usepackage{capt-of}
\usepackage{hyperref}
\usepackage{xypic}
\xyoption{pdf}
\usetheme{default}
\newcommand\cat[1]{\ensuremath{\mathcal{#1}}}
\newcommand\hm[2]{\hom(#1,#2)}
\newcommand\nat[2]{{#1}\overset{\cdot}{\rightarrow}{#2}}
\newcommand\tup[1]{\langle #1\rangle}
\newcommand\id[1]{\mathrm{id}_{#1}}
\author{jens.jensen@stfc.ac.uk \\0000-0003-4714-184X\\CC-BY 4.0}
\date{March 13, 2022}
\title{Functional Programming 2}
\hypersetup{
 pdfauthor={jens.jensen@stfc.ac.uk},
 pdftitle={Functional Programming 2},
 pdfkeywords={functional, monad, programming theory},
 pdfsubject={functional programming},
 pdfcreator={Emacs 27.1 (Org mode 9.3) then by hand}, 
 pdflang={English}}
\begin{document}

\maketitle
\begin{frame}{Outline of Talk 2}

  \begin{itemize}
    \item Previous talk (talk 1):
    \begin{itemize}
\item Introibo
\item Pure Functional Programming Principles
    \end{itemize}
  \item This talk (and probably the next):
    \begin{itemize}
    \item Mapping
    \item Labels and naming
    \item Lists
    \end{itemize}
    
  \item Advanced(ish) Topics
\item Impure Functional? Side Effects
\item Category Theory
\item Categories and Functions
\item Categories and Computation
\end{itemize}

Still written in the author's spare time!

\medskip

Very much a personal perspective, and not following any particular textbook.  Using \emph{meditations} and \emph{exercises} -- solutions to all exercises given during the talks.

\end{frame}


\section{Introibo}
\label{sec:org0edb596}

\begin{frame}{Summary of Talk 1}
  \begin{itemize}
  \item Basic Principles
    \begin{itemize}
    \item Variables are immutable
    \item Functions have no side effect
      \begin{itemize}
      \item Side effects permitted on r-values (often preferred for efficiency)
      \item In this talk we look at functions with side effects
      \end{itemize}
    \item ``Divide and conquer'' approach to problems
    \end{itemize}
  \end{itemize}
\end{frame}
\begin{frame}[fragile]{Summary of Talk 1}
\begin{verbatim}
(defun fact (k)
  "Calculate the factorial of a number"
  (unless (and (numberp k) (integerp k) (>= k 0))
    (error "Unable to take factorial of %s" k))
  (fact-1 k))
\end{verbatim}
\texttt{fact} is the entry point; it delegates to a helper function which is guaranteed to be called with a non-negative integer.  This means that (in principle) checks can be turned off for the helper, and it can be optimised to run faster:
\begin{verbatim}
(defun fact-1 (k)
  (if (zerop k) 1 (* k (fact-1 (1- k)))))
\end{verbatim}
\end{frame}
\begin{frame}[fragile]{More Mapping}
  Meditation exercise: why do mapping \emph{et al.} not work with macros and special forms?
\begin{verbatim}
(apply #'and '(nil t t))
\end{verbatim}
raises an error, though these will work in EL, but not in CL:
\begin{verbatim}
(funcall #'if t 'yes 'no)
yes
(funcall #'and t t nil)
nil
\end{verbatim}
\end{frame}

\begin{frame}[fragile]{Example -- Sorting Months}
Where does August come before July, April before January?
\begin{verbatim}
(defvar +dates+ '((10 . "Aug") (2 . "Dec") (17 . "Mar") (30 . "Apr") (4 . "Jan") (2 . "Aug")))
\end{verbatim}
Let's start with a function to calculate the month number:
\begin{verbatim}
(defun month-number (m)
  (1+ (floor
        (search m
          "JanFebMarAprMayJunJulAugSepOctNovDec")
        3)))
month-number
(month-number "Jan")
1
(month-number "Dec")
12
\end{verbatim}
This function is \emph{correct} in the sense of producing the right output given the right input, but it does have some flaws and inefficiencies (exercise) -- we shall return to these later.
\end{frame}

\begin{frame}[fragile]{Example -- Sorting Months}
Now the problem is simple: first we add the month number...
\begin{verbatim}
(mapcar (lambda (d) (cons (month-number (cdr d)) d)) +dates+)
((8 10 . "Aug") (12 2 . "Dec") (3 17 . "Mar") (4 30 . "Apr") (1 4 . "Jan") (8 2 . "Aug"))
\end{verbatim}
then we sort those
\begin{verbatim}
(mapcar #'cdr
 (sort
  (mapcar (lambda (d) (cons (month-number (cdr d)) d)) +dates+)
  (lambda (d1 d2) (or (< (first d1) (first d2))
                      (and (= (first d1) (first d2))
                           (< (second d1) (second d2)))))))
((4 . "Jan") (17 . "Mar") (30 . "Apr") (2 . "Aug") (10 . "Aug") (2 . "Dec"))
\end{verbatim}
This is what Perl calls the ``Schwarzian Transform'' -- add the order
value to the entries, sort on the order value, and then strip it off
again at the end.
\end{frame}

\begin{frame}[fragile]{Example -- Sorting Months}
Of course with this being Lisp, we can do better:
\begin{verbatim}
(sort (copy-seq +dates+)
  (lambda (d1 d2)
    (let ((m1 (month-number (cdr d1)))
          (m2 (month-number (cdr d2))))
      (or (< m1 m2) (and (= m1 m2) (< (car d1) (car d2)))))))
((4 . "Jan") (17 . "Mar") (30 . "Apr") (2 . "Aug") (10 . "Aug") (2 . "Dec"))
\end{verbatim}
Meditation: why do we need \texttt{copy-seq} here when we didn't need it before?
 Is this solution better than the Schwarzian transform?

\medskip
In CL, the \texttt{:key} parameter is used to extract the field to be sorted on, but it can also be used to calculate the order:
\begin{verbatim}
(sort (copy-seq +dates+) #'<
  :key (lambda (d) (+ (car d) (* 40 (month-number (cdr d))))))
\end{verbatim}

\end{frame}

\begin{frame}[fragile]{Example -- Magic Mapping with Apply}
From the author's AoC21, Day 9, in CL rather than EL:
\begin{verbatim}
(defun transpose (rows)
  "Transpose rows and columns, list of lists version"
  (apply #'mapcar #'list rows))
\end{verbatim}
This magic works because \texttt{apply} accepts the extra argument (\texttt{\#'list}) to pass to \texttt{mapcar}:
\begin{verbatim}
(transpose '((1 2 3) (4 5 6)))
((1 4) (2 5) (3 6))
\end{verbatim}
\end{frame}
\begin{frame}[fragile]{Advanced Maps}
  Meditate on the advanced mapping functions: this CL example is also from AoC21 Day 9 (uses \texttt{incf} to get map-indexed):
\begin{verbatim}
(defun row-troughs (row)
  "Return the locations of the \"troughs\" in a row of
   numbers, local minima"
  ;; General case: haven't thought too much about it
  (let* ((ext-row (cons most-positive-fixnum
                     (append row
                        (cons most-positive-fixnum nil))))
         (idx 0))
    (mapcan (lambda (a b c)
              (prog1 (if (and (> a b) (< b c))
                       (list idx) nil) (incf idx)))
          ext-row (cdr ext-row) (cddr ext-row))))
\end{verbatim}
\end{frame}
\begin{frame}[fragile]{Example Koan - fizzbuzz 1}
\begin{verbatim}
(defun buzz (num) (if (zerop (mod num 5)) (list 'buzz) nil))
(defun fizz (num)
  (funcall
   (if (zerop (mod num 3))
       (lambda (x) (cons 'fizz x)) #'identity)
   (buzz num)))
(fizz 2)
nil
(fizz 6)
(fizz)
(fizz 10)
(buzz)
(fizz 30)
(fizz buzz)
\end{verbatim}
Meditation: why is \texttt{fizz} defined like this?  (We'll write a cleaner version later)
\end{frame}
\begin{frame}[fragile]{Example Koan - fizzbuzz 2}
As an aside, would this work (i.e.\ without using \texttt{funcall})?
\begin{verbatim}
...
((if (zerop ...) (lambda (x) ...) #'identity) (buzz num))
\end{verbatim}

\end{frame}
\begin{frame}[fragile]{Example Koan - fizzbuzz 3}
Are these better -- and which of these is the best?
\begin{verbatim}
(defun fizz (num)
  (let ((b (buzz num)))
    (if (zerop (mod num 3)) (cons 'fizz b) b)))
\end{verbatim}
or
\begin{verbatim}
(defun fizz (num)
  (if (zerop (mod num 3)) (cons 'fizz (buzz num))
      (buzz num)))
(fizz 7)
nil
(fizz 3)
(fizz)
(fizz 5)
(buzz)
(fizz 15)
(fizz buzz)
\end{verbatim}

\end{frame}
\begin{frame}[fragile]{Example Koan - fizzbuzz 4}
Digression: We need to generate lists of integers (called iota from APL, A+ et al):
\begin{verbatim}
(defun iota (k)
  "Generate a list of integers from 1 to k" 
  (labels ((iota1 (k1)
                  (if (< k1 1) nil
                    (cons k1 (iota1 (1- k1))))))
    (nreverse (iota1 k))))
\end{verbatim}
Meditations:
\begin{itemize}
\item Ponder the use of \texttt{nreverse} -- why it is needed, why it is safe
  \begin{itemize}
  \item We shall see later how to build better \texttt{iota}s
  \end{itemize}
\item Why might the variable/function naming not be ideal?  Could we have used \texttt{k} in \texttt{iota1}?  (We will return to naming shortly)
\item \texttt{(defun fact (k) (reduce \#'* (iota k)))}
\end{itemize}

\end{frame}
\begin{frame}[fragile]{Example Koan - fizzbuzz 5}
This is not true fizzbuzz but we want to show \texttt{mapcan} in EL:
\begin{verbatim}
(defun fizzbuzz (num)
  (mapcan #'fizz (iota num)))
(fizzbuzz 15)
(fizz buzz fizz fizz buzz fizz fizz buzz)
\end{verbatim}
The point is that \texttt{fizz} can return 0, 1 or 2 results and they are merged into the result list (destructively!)
\begin{itemize}
\item \texttt{mapcan} merges lists together (destructively), so the map function can return multiple results
\item For 0 or 1 values, it may be cleaner to use \texttt{delete}, \texttt{delete-if} or \texttt{delete-if-not} (as appropriate)
  \begin{itemize}
  \item Note that logically \texttt{delete-if-not} does ``select-if''
  \item \texttt{remove}, \texttt{remove-if} and \texttt{remove-if-not} are the side-effect-free versions
  \end{itemize}
\end{itemize}
\end{frame}
\begin{frame}[fragile]{Example Koan -- FizzBuzz \$$-1$}
  Since we've started, we might as well solve FizzBuzz (with the definitions from previous slides):
\begin{verbatim}
(defun fizzbuzz-number (k)
  "Return symbol from fizzbuzzing a number, or nil"
  (let ((fb (fizz k)))
    (if (cdr fb) 'fizzbuzz  ; >1 elt
      (car fb))))

(defun fizzbuzz (k)
  "FizzBuzz from 1 to k"
  (mapcar (lambda (x) (or (fizzbuzz-number x) x))
    (iota k)))

(fizzbuzz 21)
(1 2 fizz 4 buzz fizz 7 8 fizz buzz 11 fizz ...)
\end{verbatim}
\end{frame}
\begin{frame}[fragile]{Example Koan -- FizzBuzz \$}
Incidentally, in CL if we are worried about the extra list being generated (and eventually gc'ed) in
\begin{verbatim}
(mapcar (lambda (x) ...) (iota k))
\end{verbatim}
we can reuse the list with (permitted) side effect (as the value produced by \texttt{iota} is an r-value):
\begin{verbatim}
(let ((m (iota k)))
  (map-into m (lambda (x) ...) m))
\end{verbatim}
This tells Lisp that we want to reuse \texttt{m} (on the LHS of the lambda) while mapping the values of \texttt{m} (on the RHS), but it obviously less readable than a \texttt{mapcar} and requires the \texttt{let} binding.

\medskip
Unfortunately, while EL has a function called \texttt{map-into}, it does something quite different.

\medskip
We will return to fizzbuzz in the Advanced Topics sections 10 and 13.
\end{frame}

\begin{frame}[fragile]{Advanced maps: Advanced List Iterations}
\texttt{dolist} iterates over a list.  Functionally we have done the same either with \texttt{first} and \texttt{rest} and \emph{recursion}, or with \texttt{mapcar}.  The latter is like \texttt{dolist} except it produces an output:
\begin{verbatim}
(mapcar
  (lambda (state) (when (goalp state) (throw 'found state)))
  data)
(nil nil nil nil nil ...)
\end{verbatim}
An output list is needlessly produced, as we here are calling only for side effect (more on side effects later).

\medskip
Lisp has functions which will happily do the same and discard the result of $\lambda$:
\begin{verbatim}
(map nil (lambda (state)
           (when (goalp state)
             (throw 'found state)))
  data)
\end{verbatim}
\end{frame}
\begin{frame}[fragile]{Advanced maps}
How does \texttt{map} differ from \texttt{mapcar}?  It works on \emph{sequences}:
\begin{verbatim}
(map 'list (lambda (x) (+ x 2)) [1 2 3 4])
(3 4 5 6)
(map 'vector (lambda (x) (- x 2)) '(3 4 5 6))
[1 2 3 4]
(map 'string (lambda (x) (+ x 32)) "ABCD")
"abcd"
\end{verbatim}
(the latter being EL only) but accepts \texttt{nil} as its type to discard the output.

\medskip
Of course this is cleaner if only the type change is desired:
\begin{verbatim}
(coerce (list 1 2 3 4) 'vector)
[1 2 3 4]
\end{verbatim}
\end{frame}

\begin{frame}[fragile]{Two patterns implementing \texttt{mapcar}}
We could have written EL's (simpler) \texttt{mapcar} like this:
\begin{verbatim}
(defun our-mapcar (func lst)
  (if (endp lst) nil
    (cons (funcall func (first lst))
          (our-mapcar func (rest lst)))))
our-mapcar
(our-mapcar #'sqrt '(16 4 36 5))
(4.0 2.0 6.0 2.23606797749979)
\end{verbatim}
except that it may run out of stack for a longer list (we return to this problem later, in Advanced Topics section 5)

\begin{itemize}
\item Indeed much of Lisp can be (or is) constructed from a smaller set of primitives (McCarthy)
\item As we have seen before, the \texttt{first}/\texttt{rest} (or \texttt{car}/\texttt{cdr}) pattern is very common
\item As is the \texttt{cons}/\texttt{nil} constructing the result
\item Usually this (combined) pattern is implemented with \texttt{mapcar}
\end{itemize}
\end{frame}

\begin{frame}[fragile]{Advanced maps -- sliding window}
But maps can do more advanced stuff.  Let us first return to the sliding window exercise from the first talk.  We want to sum values
\begin{verbatim}
(sum-window 3 '(1 5 4 1 3 6 2))
(10 10 8 10 11)
(sum-window 2 '(1 5 4 1 3 6 2))
(6 9 5 4 9 8)
(sum-window 4 '(1 2 3))
nil
\end{verbatim}
\end{frame}
\begin{frame}[fragile]{Sliding Window}
Let's start by writing a helper function: it is to sum variables from a list:
\begin{verbatim}
(sum-helper 4 '(2 5 1 3 6 9 1))
11
(sum-helper 1 '(2 5 1 3 6 9 1))
2
(sum-helper 0 '(2 5 1 3 6 9 1))
0
(sum-helper 5 '(1 2 3 4))
nil
\end{verbatim}
This example is Interesting because it has two stop conditions: the counter reaching zero, and the list becoming empty.

\medskip
Meditation: what is \texttt{(sum-helper 0 nil)}?  Should it be \texttt{0} or \texttt{nil}?
\end{frame}
\begin{frame}[fragile]{Sliding Window}
\begin{verbatim}
(defun sum-helper (n data)
  (cond
   ((zerop n) 0)
   ((endp data) nil)
   (t (let ((result (sum-helper (1- n) (rest data))))
        (and result (+ (first data) result))))))
\end{verbatim}
\end{frame}

\begin{frame}[fragile]{Sliding Window}
Now we can write a function to call the helper... we can't just do \texttt{mapcar} as the function would see only \texttt{first} (as with \texttt{dolist}) -- so, instead, we could almost do:
\begin{verbatim}
(defun sum-window (k data)
  (if (endp data) nil
    (cons (sum-helper k data)
          (sum-window k (cdr data)))))
(sum-window 3 '(1 5 4 1 3 6 2))
(10 10 8 10 11 nil nil)
\end{verbatim}
This works, but obviously it generates extra \texttt{nil}s at the end as it has to map across the whole list - so we could write another function to truncate them away.

\medskip
Notice the \texttt{cons}/\texttt{rest} pattern again -- \emph{nearly} like the \texttt{mapcar} pattern
\end{frame}

\begin{frame}[fragile]{Sliding Window -- Advanced maps}
The \texttt{cons}-\texttt{cdr} pattern from the previous slide is a very common functional pattern.  Lisp has a mapping function that can do the same in a single line:
\begin{verbatim}
(defun sum-window (k data)
  (maplist (lambda (d) (sum-helper k d)) data))
(10 10 8 10 11 nil nil)
\end{verbatim}
\texttt{maplist} \texttt{cdr}s across the list, \texttt{cons}ing the results:
\begin{verbatim}
(maplist #'identity '(1 2 3 4 5))
((1 2 3 4 5) (2 3 4 5) (3 4 5) (4 5) (5))
\end{verbatim}
These are successive \texttt{cdr}s of the \emph{same} list:
\begin{verbatim}
(let ((m (maplist #'identity '(1 2 3 4 5))))
  (eq (cdar m) (cadr m)))
t
\end{verbatim}
\end{frame}

\begin{frame}[fragile]{Sliding Window -- Advanced maps}
So we either do
\begin{verbatim}
(defun sum-window (k data)
  (delete nil (maplist (lambda (d) (sum-helper k d)) data)))
\end{verbatim}
and we're done; or we do
\begin{verbatim}
(defun sum-window (k data)
  (mapcon (lambda (d)
            (let ((v (sum-helper k d)))
              (and v (list v))))
          data))
(sum-window 3 '(1 5 4 1 3 6 2))
(10 10 8 10 11)
\end{verbatim}
\texttt{mapcon} is to \texttt{maplist} what \texttt{mapcan} is to \texttt{mapcar} -- it \texttt{nconc}s the results; here we use it to delete the \texttt{nil} entries.

\end{frame}
\begin{frame}[fragile]{Mapping for side effect}
This call appears to have the side effect of the lambda, but it doesn't -- why?
\begin{verbatim}
(mapcar (lambda (x) (setq x 3)) (list 1 2 3 4))
(3 3 3 3)
\end{verbatim}
but in fact it has no effect at all on the original list:
\begin{verbatim}
(let ((y (list 1 2 3 4)))
  (mapcar (lambda (x) (setq x 3)) y)
  y)
\end{verbatim}
Meditation: why are we using \texttt{list} to create the list instead of the macro \texttt{'}?
\end{frame}

\begin{frame}[fragile]{Mapping for side effect}
This works, though -- why?
\begin{verbatim}
(let ((y (list 1 2 3 4)))
  (maplist (lambda (x) (rplaca x (1+ (car x)))) y)
  y)
(2 3 4 5)
\end{verbatim}
but of course \texttt{maplist} still creates a temporary list which is also updated:
\begin{verbatim}
(let ((y (list 1 2 3 4)))
  (maplist (lambda (x) (rplaca x (1+ (car x)))) y))
(2 3 4 5)
\end{verbatim}
The lists are not \texttt{eq} (though their elements are):
\begin{verbatim}
(let ((y (list 1 2 3 4)))
  (eq y (maplist (lambda (x) (rplaca x 3)) y)))
nil
\end{verbatim}
\bigskip
This ends the review of the mapping functions!
\end{frame}
\end{document}
